\chapter*{Introduction}

A \textit{number field} is a finite extension $L/\bq$. A \textit{function field}
(over a finite field) is a finite extension of $\ffq(t)$.

Number fields have been one of the main objects of study in algebraic number
theory for many years. Function fields arise naturally in algebraic geometry, as
the rings of global functions on smooth projective curves.

The analogy between these two classes of fields is deep and fascinating: in
particular, there are instances of results being proven on one side and
subsequently being transported to the other side. For instance, the classical
subject of class field theory has an analog for function fields, called
\textit{geometric class field theory}.

\section*{Acknowledgements}
This paper was supervised by David Roe, who also taught me a lot of number
theory in my time at Mathcamp 2016, and encouraged me to work on this project.
His guidance has been invaluable, and I have learned a lot from the fruitful
conversations we have had.

I am indebted to Clifton Cunningham for introducing me to $p$-adic fields and
ad\`eles in a class on the character group of $\bq$, for referring me to
\cite{artinwhaples}, the paper which I have treated here, and for convincing me
to work towards learning arithmetic geometry.

Thanks to Frank Dai for going through a draft of this paper.

\section*{Notation and conventions}
\begin{enumerate}
\item For an infinite set, a property holds for \textit{almost all} of its
  elements if it is satisfied by all but a finite number of elements.
\item Unless otherwise noted, $p$ is an (integer) prime. In the same vein, $q=
  p^n$ for some prime $p$ and positive power $n$.
\item We will write $G_K$ for the absolute Galois group
  \[ G_K := \gk \]
  for $K$ a (usually number) field.
\item We use the computer science-inspired notation
  \[ A := B \]
  to indicate that some piece of notation $A$ is being defined to mean $B$.
\end{enumerate}
