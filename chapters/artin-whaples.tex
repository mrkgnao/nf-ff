\chapter{The Artin-Whaples characterization}
\section{Introduction}
\label{sec:orgheadline1}

A striking piece of evidence in favor of our hypothesis that number fields and
function fields are more similar than one might expect is given by
\cite{artinwhaples}, which proves the following theorem:

\begin{thm}[Main theorem of \cite{artinwhaples}]
  \label{thm:maintheorem}
  If a field satisfies the valuation product formula, and if one of those
  valuations is of a suitable type, then it is forced to be either a number
  field or a function field.
\end{thm}

We will now examine the proof of the following theorem, essentially following
the original in its development of the material.

\section{Places and valuations}
\label{sec:orgheadline6}

\subsection{Valuations}
\label{sec:valuations}

A valuation on a field is a way to assign a ``size'' to its elements in a way
that fits our usual expectations of how such functions should behave. For
instance, we have the valuation
\[ |\cdot| : \bc \to \br \]

which is defined by the mapping

\[ |a + ib| \mapsto \sqrt{a^2 + b^2} \]

for $a+ib\in\bc$.

The properties this satisfies (nonnegativity, the triangle inequality, and so
on) are abstracted by the following definition:

\begin{defn}
  Let $k$ be a field. A function $|\cdot|: k\to\br$ is called a
  \textit{valuation} if it satisfies the following properties:\footcite[section
  1]{artinwhaples}
  \begin{enumerate}
  \item $|\alpha| = 0 \iff \alpha = 0$
  \item $\im |\cdot|\subset \rgt$
  \item $|\alpha\beta| = |\alpha||\beta|$
  \item $|\alpha + \beta| \leq |\alpha| + |\beta|$
  \end{enumerate}
\end{defn}

If a valuation satisfies the following, it is called \textit{nonarchimedean}
(and \textit{archimedean} otherwise):

\begin{enumerate}
\item[3'] $|\alpha + \beta| \leq \max(|\alpha|,|\beta|)$
\end{enumerate}

\subsection{Motivation for places}
\label{sec:whatisaplace}
When working with number fields other than $\bq$, we find that there are ``more
primes'' than we might expect. In a naive sense, of course, this is true: for
instance, we have primes like $(1+i)$ in $\bq(i)$.

More generally, we can look at a prime ideal in the ring of integers and
consider the valuation it gives rise to.


For instance, $(3)\subset\zi$ gives us
the valuation

\[ \val x 3 = \val x {(3)} = 3^{-\nu_3(x)} \]

with, e.g. $ \val {36} 3 = 3^{-2} $.

We might then decide to consider the valuations themselves as the fundamental
objects. This is very useful: considering valuations allows us to recover a lot
of data, like the prime ideals of the ring of integers, in a purely
\textit{field-theoretic} way, instead of having to worry about integral closures
and so on.\footnote{I realised the importance of this thanks to \cite{placemse}.}

Of course, this on its own is not very useful, since there are many,
many more possible valuations than there can be ``generalized primes'' (however one
wishes to define that).

\subsection{The equivalence relation on valuations}

The solution is to define a notion of \textit{equivalence} for valuations. One
way to do it is by noticing the following:

\begin{thm}
  Every valuation on a field induces a metric on it.
\end{thm}

A metric defines a metric space structure, and hence a topological space
structure, on the field. We can now say that

\begin{defn}
  Two valuations $\val \cdot 1$ and $\val\cdot2$ are \textit{equivalent} if
  they determine identical topological space structures on the field.
\end{defn}

Another way to do it is as follows: first, notice that raising an absolute value
to any power less than $1$ gives rise to another absolute value. We can, hence,
define two absolute values to be equivalent if there is some power $c\in(0,1)$
for which \[|\cdot|_1 = |\cdot|_2^c\].

These two definitions of \textit{equivalent} are actually equivalent!

\begin{defn}
  An equivalence class of valuations is called a \textit{prime place}, or simply
  a \textit{place} for short.
\end{defn}

The (nontrivial) places of $\bq$ can be classified as follows:

\begin{itemize}
\item the nonarchimedean ones, which correspond to $p$-adic valuations and hence
  to prime ideals $(p)$ in the ring of integers $\bz$ of $\bq$ (and thus to the
  $p$-adic fields arising as completions of $\bq$ under these valuations)
\item the archimedean ones, which correspond to embeddings of $\bq$ into $\br$
  or $\bc$ (``real'' and ``complex'' embeddings)
\end{itemize}

In general, a field may have many archimedean places, corresponding to different
embeddings into $\br$ or $\bc$.

\section{The proof}
\subsection{A finite set of valuations}
\label{sec:orgheadline3}

We now look at a \textit{finite} set of (nontrivial, inequivalent) valuations
$|\cdot|_i$. We have the following\footcite[p. 471--2]{artinwhaples} results:

\begin{lemma}
  If \(|\cdot|_1\) and \(|\cdot|_2\) are two inequivalent valuations, there is
  some \(\gamma\) such that \[|\gamma|_1 < 1 \text{ and } |\gamma|_2 > 1.\]
\end{lemma}

\begin{lemma}
  If \(|\cdot|_i\) are inequivalent, there is some \(\alpha\) such
  that \[|\alpha|_1 > 1\text{ and }|\alpha|_{i>1} < 1.\]
\end{lemma}

\begin{lemma}
  If \(|\cdot|_i\) are inequivalent, for every \(\epsilon>0\), there is an
  \(\alpha\) such that \[|\alpha-1|_1 \leq 1 \text{ and } |\alpha|_{\nu>1}\leq
    1.\]
\end{lemma}

\begin{thm}[Approximation theorem]
  \label{eq:approximation}
  Given pairs \((|\cdot|_i, \alpha_i)\), with the \(|\cdot|_i\) inequivalent,
  then for every \(\epsilon>0\) there is some \(\alpha\) with
  \[|\alpha - \alpha_i|_i < \epsilon.\]
\end{thm}

\begin{cor}
  \label{eq:corollary}
  If \(|\cdot|_i\) are nontrivial and inequivalent, then any identity of the
  form \[\prod |\alpha|_{i}^{\nu_i} = 1\] with \(0\neq \alpha\in k\) implies
  that the \(\nu_i\) are all \(0\).
\end{cor}

Theorem \ref{eq:corollary} ``precludes the possibility that a finite number of
valuations can ever be interrelated''.

\section{The product formula}
\label{sec:productformula}

We now come to the product formula referred to in \ref{thm:maintheorem}. Recall
that it states that a field satisfying two particular axioms with respect to the
valuations defined on it must either be a number field or a function field.

We will now look at the two axioms of the theorem.

\subsection{The valuation product formula}

\begin{axiom}
\label{eq:axiom1}
  There is a set \(M\) of pairs \((\pp, |\cdot|_\pp)\) such that, for any
  \(0\neq \alpha \in k\),
  \begin{itemize}
  \item \(\val\alpha\pp = 1\) for almost all \(\pp\)
  \item Extending the product over all primes,
    \[ \prod_\pp \val\alpha\pp = 1 \]
  \end{itemize}
  For instance, for \(6\in\bq\), the product looks like
  \[ \val6{(0)} \cdot\val6{(2)}\cdot\val6{(3)} = 6\cdot 2^{-1} \cdot 3^{-1} =
    1 \]
\end{axiom}

We have mentioned earlier that to any prime $\pp$, we can associate infinitely
many equivalent valuations on $k$. This axiom allows us to choose one particular
valuation as a distinguished representative of its equivalence class.

\subsection{Id\`eles}
\label{sec:orgheadline19}
We associate to \(M\) a space of vectors $V_M$, whose elements are vectors of
the form

\[ v={ (v_\pp) }_\pp, \text{where } v_\pp \in k_\pp. \]

We will simplify notation by writing \(\val v\pp\) for \(\val{v_\pp}\pp\).

\begin{defn}
  A vector $(v_\pp)$ of this form is an \textit{id\`ele} if
  \begin{itemize}
  \item \(v_\pp\neq 0\) for all \(\pp\)
  \item \(v_\pp = 1\) for almost all \(\pp\)
  \end{itemize}
\end{defn}

There is a natural embedding \(k\hookrightarrow V_M\), reminiscent of diagonal embeddings: writing \(i_\pp\) for the inclusion \(k\to k_\pp\),
\[\alpha\mapsto{ (i_\pp(\alpha)) }_\pp\]

\subsection{The volume of an id\`ele}
\label{sec:orgheadline12}

For each id\`ele \(\fA\), define
\[V(\fA) = \prod_\pp \val\fA\pp \]
This function is obviously multiplicative.

\begin{enumerate}
\item For elements $\alpha$ coming from \(k\) via the embedding, notice that we have, via
  the product formula \ref{sec:productformula},

  \[V(\alpha) = \prod_\pp \val\fA\pp = 1.\]

  This gives \[V(\alpha\fA) = V(\alpha)V(\fA) = V(\fA). \]

\item A map \(\pp\mapsto x_\pp \in\br\), with \(x_\pp = 1\) for almost all
  \(\pp\), gives a set of vectors \[\val \fC\pp \leq \val\fA\pp\] which we call
  the parallelotope with dimensions \(x_\pp\) or even simply \(x\). In this
  case, the number $V(\fA)$ can be thought of as the ``volume'' of the
  parallelotope.

\end{enumerate}

\section{Other notions}
\subsection{The order of a set}
\label{sec:order}

\cite{artinwhaples} defines a notion of \textit{order},
stating that it
\begin{quotation}
[unites] different types of fields\footcite[p. 474]{artinwhaples}
\end{quotation}

although the definition itself is a a construction I have little motivation for.

Given a ``field of discourse'' $k$, the order of a set of elements is defined as follows:
\begin{enumerate}
\item If \(k\) has an archimedean valuation, then the order of a set is its the
  number of elements.
\item If all the valuations of $k$ are nonarchimedean, there is some field of constants \(k_0
\subset k\). (For instance, if $k = \bff_q(\sqrt t)$, $k_0 = \bff_q$.) The order of a set is then defined to be \(q^s\), where we define
\(q\) and \(s\) as follows:
  \begin{enumerate}
  \item If \(k_0\) is finite, \(q\) is its number of elements. Otherwise, \(q\) is an arbitrary fixed number greater than \(1\).
  \item \(s\) is the number of elements in the set that are linearly independent over \(k_0\).
  \end{enumerate}
\end{enumerate}

\subsection{The \(M\)-function}
\label{sec:orgheadline17}
The order of a set of elements contained in the parallelotope of size \(\fA\) will be denoted \(M(\fA)\).
Note that, for nonzero \(\theta\in k\), \(M(\theta\fA)=M(\fA)\) since multiplying by \(\theta\) changes the parallelotope of size \(\fA\) into the parallelotope of size \(\theta\fA\) and does not change the order.

\subsection{The ring of \(\pp\)-integers}
\label{sec:orgheadline18}
The set of elements \(\alpha\in k\) for which \(\val\alpha\pp \leq 1\) forms a ring, which we denote \(\op\).
The subset of \(\op\) with \(\val\alpha\pp < 1\) forms an ideal in this ring, which, by abuse of notation, is also denoted \(\pp\). Now we have a quotient field \(\op/\pp\), and so on. The \emph{order} of this field is called the norm \(N\pp\) of \(\pp\). For instance, if there is a constant field \(k_0
\subseteq k^{\pp} = \op/\pp\), we have
\[ N\pp = {(\# k_0)}^{[k^\pp : k_0]} \]

\section{Axiom 2}
\label{sec:orgheadline20}
The set \(M\) of \ref{subsec:axiom1} contains at least one prime \(\qq\), which is either
\begin{itemize}
\item discrete, with a finite quotient field of finite order \(N\qq\)
\item archimedean, with \(k_\qq=\br\) or \(\bc\)
\end{itemize}

Note that, by Ostrowski's theorem, the second part of the condition on
archimedean places is superfluous.

\subsection{Another valuation}
\label{sec:orgheadline21}
For \(\alpha\neq 0\), define a valuation as follows:
\begin{itemize}
\item For \(\pp \not | \infty\) set
  \[ { ||\alpha|| }_\pp = \frac1{ N\pp^\nu } \]
  where \(\nu = \ord_\pp(\alpha)\).
\item If \(k=\br\), \(||\cdot||_\pp\) is defined to be the standard absolute value.
\item If \(k=\bc\), \(||\cdot||_\pp\) is set to be the squared absolute value.
\end{itemize}

\subsection{Number fields and function fields work}
We next come to the following theorem, which tells us that number fields and
function fields satisfy the two axioms stated above:

\begin{thm}
  We can construct \(M\) such that both \ref{subsec:axiom1} and
  \ref{sec:orgheadline20} hold for the following fields:
  \begin{itemize}
  \item a number field, i.e., a finite extension \(K/\bq\)
  \item a field of algebraic functions over any field \(k_1\) (that is, a finite
    extension \(K/k_1(z)\) with \(z\) transcendental \(/k_1\))
  \end{itemize}
In the second case, the field of constants $k_0 \subset k$ consists of all
elements of $k$ algebraic over $k_1$.
\end{thm}

The proof of this is done through a sequence of lemmas.

\begin{lemma}
 Let $k$ be a field for which \ref{eq:axiom1} holds, and $F$ a subfield that
 does not exclusively comprise constants of $k$. Let $\fn$ be the set of
 nontrivial divisors (???) $p$ of $F$ that are divisible by some $\pp$ of $\fm$.
 Then \ref{eq:axiom1} holds in $F$ for $\fn$.
\end{lemma}

\begin{lemma}
Let $k$ be a field for which \ref{eq:axiom1} holds and $K/k$ a finite algebraic
extension. Let $\fn$ be the set of all divisors $\fp$ of $K$ that divide some
$\pp$ of $\fm$. Then \ref{eq:axiom1} holds in $K$ for some subset $\fn' \subset \fn$.
\end{lemma}

In fact, $\fn' = \fn$, but this is postponed to the next section.

\begin{lemma}
\ref{eq:axiom1} holds in the case of $\bq$ and that of $K=k_1(z)$. The set $\fm$
of valuations is the set of all valuations in the case of $\bq$, and the set of
valuations trivial on $k_1$ in the latter case. The product formula is of the
form

\[ \prod_p {||a||}_p = 1 \]

or a power, and there is no other relation between these valuations.

\end{lemma}

From the last two lemmas, we see that \ref{eq:axiom1} holds for the fields of
(??? thm2). The fact that all valuations of $\fm$ satisfy \ref{eq:axiom2}
follows from the fact that this is true in $\bq$ and hence in a finite extension
$k/\bq$.

It  remains  to  prove  the  statement  about  the  field  fe 0   of  constants.

If  p  is  trivial  on   k   it  is  also  trivial  on  an  algebraic  extension  of    ki.    Hence  we  need  only  show  that  any  constant   c   of   ko  is  algebraic  with   respect  to   ki.   If  on  the  contrary   c  were  transcendental  with  respect   to   ki   then  from  the  equation   c  satisfied  with  respect  to   ki(z)   it  follows   that   z   would  be  algebraic  with  respect  to   ki(c).    Since   ki(c)    is  in   ko,   this would  mean that   z  is in   ko.  So all  of   k  would be in fe 0 , contradicting  the  fact  that  no  p of  9JÎ is trivial  on   k.

\section{Characterizing fields by the product formula}
\label{sec:orgheadline28}

\subsection{(Main) theorem 3}
\label{sec:orgheadline27}
If a field satisfies \ref{subsec:axiom1} and \ref{sec:orgheadline20}, it is of one of the two types in \ref{sec:orgheadline25}. Furthermore, \ref{sec:orgheadline20} is satisfied for every place \(\pp\).

\section{Parallelotopes}
\label{sec:orgheadline31}

\subsection{Theorem 4}
\label{sec:orgheadline29}
There are positive \(C, D\) such that for all id\`eles \(\fA\) we have
\[ CV(\fA) < M(\fA) \leq \max(1, DV(\fA)) \]


\subsection{Definitions}
\label{sec:orgheadline30}
Let \(U\) be the multiplicative group of ``absolute units'', that is, \(x\in k\) is in \(U\) if \({||x||}_\pp = 1\) for all \(\pp\).
\begin{itemize}
\item If there is a constant field \(k_0\), \(U=\ut {k_0}\).
\item ``In case order means number of elements, \(U\) must be a finite group since it is contained in the parallelotope of size \(1\), so \(U\) consists of all roots of unity in \(k\).
\end{itemize}
Now select a finite set \(S\) of primes that contains all the archimedean primes. By \(\fA_S\) we mean the id\`eles \(\fA\) such that \(\vval{ \fA } = 1\) for all \(\pp\not\in S\). As one might expect, \(e_\pp\in k\) which belong to \(\fA_S\) are called \(S\)-units.
\end{document}
