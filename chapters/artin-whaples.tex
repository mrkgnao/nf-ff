\chapter{The Artin-Whaples characterization}
\section{Introduction}
\label{sec:orgheadline1}

A striking piece of evidence in favor of our hypothesis that number fields and
function fields are more similar than one might expect is given by
\cite{artinwhaples}, which proves the following theorem:

\begin{thm}[Main theorem of \cite{artinwhaples}]
  \label{thm:maintheorem}
  If a field satisfies the valuation product formula, and if one of those
  valuations is of a suitable type, then it is either a number field or a
  function field.
\end{thm}

We will follow the proof of one direction of the theorem, setting out the two
axioms and showing that number fields and function fields satisfy them.

\section{Places and valuations}
\label{sec:orgheadline6}

\subsection{Valuations}
\label{sec:valuations}

A valuation on a field is a way to assign a ``size'' to its elements in a way
that fits our usual expectations of how such functions should behave. For
instance, we have the valuation
\[ |\cdot| : \bc \to \br \]

which is defined by the mapping

\[ |a + ib| \mapsto \sqrt{a^2 + b^2} \]

for $a+ib\in\bc$.

The properties this satisfies (nonnegativity, the triangle inequality, and so
on) are abstracted by the following definition:

\begin{defn}
  Let $k$ be a field. A function $|\cdot|: k\to\br$ is called a
  \textit{valuation} if it satisfies the following properties:\footcite[section
  1]{artinwhaples}
  \begin{enumerate}
  \item $|\alpha| = 0 \iff \alpha = 0$
  \item $\im |\cdot|\subset \rgt$
  \item $|\alpha\beta| = |\alpha||\beta|$
  \item $|\alpha + \beta| \leq |\alpha| + |\beta|$
  \end{enumerate}
\end{defn}

If a valuation satisfies the following, it is called \textit{nonarchimedean}
(and \textit{archimedean} otherwise):

\begin{enumerate}
\item[3'] $|\alpha + \beta| \leq \max(|\alpha|,|\beta|)$
\end{enumerate}

\subsection{Motivation for places}
\label{sec:whatisaplace}
When working with number fields other than $\bq$, we find that there are ``more
primes'' than we might expect. In a naive sense, of course, this is true: for
instance, we have primes like $(1+i)$ in $\bq(i)$.

More generally, we can look at a prime ideal in the ring of integers and
consider the valuation it gives rise to.


For instance, $(3)\subset\zi$ gives us
the valuation

\[ \val x 3 = \val x {(3)} = 3^{-\nu_3(x)} \]

with, e.g. $ \val {36} 3 = 3^{-2} $.

We might then decide to consider the valuations themselves as the fundamental
objects. This is very useful: considering valuations allows us to recover a lot
of data, like the prime ideals of the ring of integers, in a purely
\textit{field-theoretic} way, instead of having to worry about integral closures
and so on.\footnote{I realised the importance of this thanks to \cite{placemse}.}

Of course, this on its own is not very useful, since there are many,
many more possible valuations than there can be ``generalized primes'' (however one
wishes to define that).

\subsection{The equivalence relation on valuations}

The solution is to define a notion of \textit{equivalence} for valuations. One
way to do it is by noticing the following:

\begin{thm}
  Every valuation on a field induces a metric on it.
\end{thm}

A metric defines a metric space structure, and hence a topological space
structure, on the field. We can now say that

\begin{defn}
  Two valuations $\val \cdot 1$ and $\val\cdot2$ are \textit{equivalent} if
  they determine identical topological space structures on the field.
\end{defn}

Another way to do it is as follows: first, notice that raising an absolute value
to any power less than $1$ gives rise to another absolute value. We can, hence,
define two absolute values to be equivalent if there is some power $c\in(0,1)$
for which \[|\cdot|_1 = |\cdot|_2^c\].

These two definitions of \textit{equivalent} are actually equivalent!

\begin{defn}
  An equivalence class of valuations is called a \textit{prime place}, or simply
  a \textit{place} for short.
\end{defn}

The (nontrivial) places of $\bq$ can be classified as follows:

\begin{itemize}
\item the nonarchimedean ones, which correspond to $p$-adic valuations and hence
  to prime ideals $(p)$ in the ring of integers $\bz$ of $\bq$ (and thus to the
  $p$-adic fields arising as completions of $\bq$ under these valuations)
\item the archimedean ones, which correspond to embeddings of $\bq$ into $\br$
  or $\bc$ (``real'' and ``complex'' embeddings)
\end{itemize}

In general, a field may have many archimedean places, corresponding to different
embeddings into $\br$ or $\bc$.

\section{The proof}
\subsection{A finite set of valuations}
\label{sec:orgheadline3}

We now look at a \textit{finite} set of (nontrivial, inequivalent) valuations
$|\cdot|_i$. We have the following\footcite[p. 471--2]{artinwhaples} sequence of
lemmas (the proofs are somewhat ``analytic'' in nature and might be skipped on a
first reading), culminating in a very useful approximation theorem.

\begin{lemma}
  If $|\cdot|_1$ and $|\cdot|_2$ are two inequivalent valuations, there is
  some $\gamma$ such that \[|\gamma|_1 < 1 \text{ and } |\gamma|_2 > 1.\]
\end{lemma}

\begin{proof}
  Equivalent valuations can each be expressed as a power of the other. For
  inequivalence, there must be some $\alpha$ and $\beta$ such that

  \begin{align*}
    \val \alpha 1 &<    1 \leq \val \alpha 2 \\
    \val \beta 1  &\geq 1 >    \val \beta 2
  \end{align*}

  We can then set $\gamma = \alpha/\beta$.
\end{proof}

\begin{lemma}
  If $\val \cdot 1, \val \cdot 2, \ldots, \val \cdot n$ are nontrivial and
  inequivalent, there is some $\alpha$ such that

  \[\val \alpha 1 > 1 \]

  and, for all $i > 1$,

  \[ \val \alpha i < 1.\]
\end{lemma}

This means that we can always find some $\alpha$ that distinguishes a particular
valuation in a set of mutually inequivalent ones.

\begin{proof}
  This is just an extension of the previous lemma, which can be considered to be
  the $n=2$ case of this. We use induction on $n$, assuming we have found some
  $\gamma$ such that

  \begin{align*}
    \val \gamma 1 &> 1\\
    \val \gamma i &< 1, 1 < i \leq n - 1
  \end{align*}

  Now we apply the previous lemma to the set of valuations $\{ \val \cdot 1,
  \val \cdot n \}$ to get a $\delta$ such that $\val \delta 1 > 1$ and $\val
  \delta n < 1$. There are two possible cases:
  \begin{enumerate}
  \item $\val \gamma n \leq 1$. In this case, set $\alpha = \gamma^r \delta$.
    Then

    \[ \val \alpha 1 = \val {\gamma^r} 1 \val \delta 1 > 1 \]

    by construction. In addition, for $r$ sufficiently large,

    \[ \val \alpha i < 1 \]

    for $2\leq i \leq n$, since whatever the valuation of $\delta$ at the other
    primes, the factor of $\val \gamma i ^r$ (which is known to be less than
    $1$, by hypothesis) will eventually outweigh the factor of $\val \delta i$
    for $r$ large enough.

  \item If $\val \gamma n > 1$, set

    \[ \alpha = \frac{\gamma^r}{\gamma^r + 1} \delta \]

    so that we have

    \[ \val \alpha i = \frac{\val \gamma i ^r \val \delta i}{\val {\gamma^r + 1}
        i} \leq \frac {\val \gamma i ^r} {1 - \val \gamma i ^r} \val \delta i
    \]

    for $2 \leq i \leq n - 1$, and

    \[
      \val \alpha n \leq \frac{\val \gamma n ^r} {\val \gamma n ^r - 1} \val
      \delta n.
    \]

    For $r$ large enough, $\val \alpha i < 1$ (since the numerator goes to $0$
    as $r\to\infty$) for $2\leq i\leq n-1$.

    Note that

    \[ \lim_{r\to\infty} \frac{\val \gamma n ^r} {\val \gamma n ^r - 1} = 1 \]

    so $\val \alpha n < 1$ since $\val \gamma n < 1$ by hypothesis. Hence this
    satisfies the conditions of the theorem for $2\leq i \leq n$.

    For the case $i = 1$, observe that

    \[ \val \alpha 1 \geq \frac {\val \gamma 1 ^r} {1 + \val \gamma 1 ^r} \val
      \delta 1 \]

    so choosing a sufficiently large $r$ gives us $\val \alpha 1 > 1$.

  \end{enumerate}
\end{proof}

\begin{lemma}
  If $|\cdot|_i$ are inequivalent, for every $\epsilon>0$, there is an $\alpha$
  such that

  \[ \val {\alpha - 1} 1 \leq \epsilon \]

  and

  \[ \val \alpha i \leq \epsilon \text{ for }i>1. \]

\end{lemma}

\begin{proof}
 Choose an appropriate $\gamma$ according to the previous lemma, so that $\val
 \beta 1 > 1$ and $\val \beta i < 1$ for $i > 1$. Set

 \[ \alpha = \frac {\beta^r} {1 + \beta^r} \]

 Then

 \[ \val {\alpha - 1} 1 = \frac 1 {\val {1 + \beta^r} 1} \leq \frac 1 {\val
     \beta 1 ^r}\footnote{The $\val \cdot 1$ in the denominator is just a
     $|\cdot|$ in the original paper, which I suspect is a typo since $|\cdot|$
     hasn't been defined to mean anything at this point.} \leq \epsilon \]

 for $r$ sufficiently large, while, for $i>1$, we have

 \[ \val \alpha i = \frac {\val \beta i ^r} {\val {1 + \beta^r} i} \leq \frac
   {\val \beta i ^r} {1 - \val \beta i ^r} \leq \epsilon \]

 (again, for $r$ sufficiently large).

\end{proof}

\begin{thm}[Approximation theorem]
  \label{eq:approximation}

  Given pairs $(|\cdot|_i, \alpha_i)$, with the $|\cdot|_i$ inequivalent,
  then for every $\epsilon>0$ there is some $\alpha$ with
  \[|\alpha - \alpha_i|_i < \epsilon.\]
\end{thm}

\begin{proof}
 asdf
\end{proof}

\begin{cor}
  \label{eq:corollary}
  If $|\cdot|_i$ are nontrivial and inequivalent, then any identity of the
  form \[\prod |\alpha|_{i}^{\nu_i} = 1\] with $0\neq \alpha\in k$ implies
  that the $\nu_i$ are all $0$.
\end{cor}

Theorem \ref{eq:corollary} ``precludes the possibility that a finite number of
valuations can ever be interrelated''. Nevertheless, an infinite number of
valuations \textit{can} be interrelated in such a way -- and this, claims
\cite{artinwhaples}, characterizes number fields and function fields.

\section{The product formula}
\label{sec:productformula}

We now come to the product formula referred to in \ref{thm:maintheorem}. Recall
that it states that a field satisfying two particular axioms with respect to the
valuations defined on it must either be a number field or a function field.

We will now look at the two axioms of the theorem.

\subsection{The valuation product formula}

\begin{axiom}
  \label{eq:axiom1}
  There is a set $M$ of pairs $(\pp, |\cdot|_\pp)$ such that, for any
  $0\neq \alpha \in k$,
  \begin{itemize}
  \item $\val\alpha\pp = 1$ for almost all $\pp$
  \item Extending the product over all primes,
    \[ \prod_\pp \val\alpha\pp = 1 \]
  \end{itemize}
  For instance, for $6\in\bq$, the product looks like
  \[ \val6{(0)} \cdot\val6{(2)}\cdot\val6{(3)} = 6\cdot 2^{-1} \cdot 3^{-1} =
    1 \]
\end{axiom}

Notice that, since $\val{1+1}{(\pp)} > 1$ for archimedean $\pp$, $\M$ can only
contain finitely many archimedean places.

We have mentioned earlier that to any prime $\pp$, we can associate infinitely
many equivalent valuations on $k$. This axiom allows us to choose one particular
valuation as a distinguished representative of its equivalence class.

\subsection{Id\`eles}
\label{sec:orgheadline19}
Consider a set $M$ defined as above. We define a vector space $V_M$, whose
elements are vectors of the form

\[ v={ (v_\pp) }_\pp, \text{where } v_\pp \in k_\pp. \]

We will simplify notation by writing $\val v\pp$ for $\val{v_\pp}\pp$.

\begin{defn}
  A vector $(v_\pp)$ of this form is an \textit{id\`ele} if
  \begin{itemize}
  \item $v_\pp\neq 0$ for all $\pp$
  \item $v_\pp = 1$ for almost all $\pp$
  \end{itemize}
\end{defn}

There is a natural embedding $k\hookrightarrow V_M$, which may very well be
called a ``diagonal embedding'' (similarly to how $\bq$ has a natural inclusion
into $\adq$). Writing $i_\pp$ for the inclusion $k\to k_\pp$, this embedding
is given by

\[\alpha\mapsto{ (i_\pp(\alpha)) }_\pp\]

\subsection{The volume of an id\`ele}
\label{sec:orgheadline12}

For each id\`ele $\fA$, define
\[V(\fA) = \prod_\pp \val\fA\pp \]

This function is obviously multiplicative, the product of the two (possibly) infinite
products being well-defined since an id\`ele has valuation $1$ at almost every
place. In this sense, id\`eles can be considered ``measurable'' elements of $V_M$.

\begin{enumerate}
\item For elements $\alpha$ coming from $k$ via the embedding, notice that we have, via
  the product formula \ref{sec:productformula},

  \[V(\alpha) = \prod_\pp \val\fA\pp = 1.\]

  This gives \[V(\alpha\fA) = V(\alpha)V(\fA) = V(\fA). \]

\item A map $\pp\mapsto x_\pp \in\br$, with $x_\pp = 1$ for almost all
  $\pp$, gives a set of vectors with the $\pp$-component given by all $\fC$
  such that

  \[\val \fC\pp \leq \val\fA\pp\]

  which we call the parallelotope with dimensions $x_\pp$ or even simply $x$. In this
  case, the number $V(\fA)$ can be thought of as the ``volume'' of the
  parallelotope.

  It is later shown that all valuations are either archimedean or discrete (in
  the sense defined in \textbf{TODO ADD LINK}). If this is true, then for any
  collection of elements $x_\pp \in \br$, we can find some $\alpha_\pp \in
  \k_\pp$ with the maximal possible valuation not greater than $x_\pp$. Hence,
  we can, without loss of generality, start with an id\`ele $\fA$ and construct
  all vectors $\fC$ satisfying

  \[ \val \fC \pp \leq \val \fA \pp \]

\end{enumerate}

\section{Other notions}
\subsection{The order of a set}
\label{sec:order}

\cite{artinwhaples} defines a notion of \textit{order},
stating that it
\begin{quotation}
  [unites] different types of fields\footcite[p. 474]{artinwhaples}
\end{quotation}

although the definition itself is a a construction I have little motivation for.
This is mainly to be able to define a notion of ``size'' for the set of elements
$\alpha \in k$ that are contained in a given $\fA$-parallelotope (where we are
transparently considering elements of $k$ as elements of $V_M$ using the
diagonal embedding), which gives us some information about $\fA$.

\begin{defn}
  Given a ``field of discourse'' $k$, the order $O(S)$ of a set $S$ of elements
  is defined as follows:
  \begin{enumerate}
  \item If $k$ has an archimedean valuation, then the order of a set is its the
    number of elements.
  \item If all the valuations of $k$ are nonarchimedean, there is some field of
    constants $k_0 \subset k$. (For instance, if $k = \bff_q(\sqrt t)$, $k_0 =
    \bff_q$.) The order of a set is then defined to be $q^s$, where we define
    $q$ and $s$ as follows:
    \begin{enumerate}
    \item If $k_0$ is finite, $q$ is its number of elements. Otherwise, $q$ is
      an arbitrary fixed number greater than $1$.
    \item $s$ is the number of elements in the set that are linearly independent
      over $k_0$.
    \end{enumerate}
  \end{enumerate}
\end{defn}

If $k_0$ has $q$ elements, and our set $S$ is closed under addition and under
$k_0$-multiplication (so it forms a $k_0$-vector space) then $O(S) = q^s$ is
just the number of elements in $S$.

\subsection{The $M$-function}
\label{sec:orgheadline17}
The order of a set of elements contained in the parallelotope of size $\fA$ will
be denoted $M(\fA)$.

Note that, for nonzero $\theta\in k$, $M(\theta\fA)=M(\fA)$.
Indeed, multiplying by $\theta$ changes the parallelotope of size $\fA$ into the
parallelotope of size $\theta\fA$. This does not change the order (this can be
seen by looking at the cases in the definition of order, noting that linear
independence is not affected if all elements of a subset of a vector space are
multiplied by a field element).

\subsection{The ring of $\pp$-integers}
\label{sec:orgheadline18}
The set of elements $\alpha\in k$ for which $\val\alpha\pp \leq 1$ forms a
ring, which we denote $\op$ (the \textit{ring of $\pp$-integers}).

In the case of, say, $k = \bq$, the ring of $(2)$-integers is, as one might
expect, $\bz_2$ (and so on).

The subset of $\op$ with $\val\alpha\pp < 1$ forms an ideal in this ring, which,
by abuse of notation, is also denoted $\pp$. Now we have a quotient field
$\op/\pp$, and so on. The \emph{order} of this field, if finite, is called the
norm $\Nm\pp$ of $\pp$.

For instance, if there is a constant field $k_0 \subseteq k^{\pp} = \op/\pp$, we
have

\[ \Nm\pp = q^f \]

where we put $f = {[k^\pp : k_0]}$ and $q$ is the order of $k_0$, as before.

\section{Axiom 2}
\label{sec:orgheadline20}
The set $M$ of \ref{subsec:axiom1} contains at least one prime $\qq$, which
is of one of the following types:
\begin{itemize}
\item discrete, with a quotient field of finite order $\Nm\qq$
\item archimedean, with $k_\qq=\br$ or $\bc$
\end{itemize}

Note that, by Ostrowski's theorem, the second part of the condition on
archimedean places is superfluous.

\subsection{Another valuation}
\label{sec:orgheadline21}
For $\alpha\neq 0$, define a valuation as follows:
\begin{itemize}
\item For $\pp \not | \infty$ set
  \[ { ||\alpha|| }_\pp = \frac1{ \Nm\pp^\nu } \]
  where $\nu = \val \alpha \pp$.
\item If $k=\br$, $||\cdot||_\pp$ is defined to be the standard absolute value.
\item If $k=\bc$, $||\cdot||_\pp$ is set to be the square of the usual absolute
  value on $\bc$. This is not a valuation in the usual sense!
\end{itemize}
This is called the \textit{normed valuation at $\pp$}.

\subsection{Number fields and function fields work}
We next come to the following theorem, which tells us that number fields and
function fields satisfy the two axioms stated above:

\begin{thm}
  We can construct $M$ such that both \ref{eq:axiom1} and
  \ref{sec:orgheadline20} hold for the following fields:
  \begin{itemize}
  \item a number field, i.e., a finite extension $K/\bq$
  \item a field of algebraic functions over any field $k_1$ (that is, a finite
    extension $K/k_1(z)$ with $z$ transcendental over $k_1$)
  \end{itemize}
  In the second case, the field of constants $k_0 \subset k$ consists of all
  elements of $k$ algebraic over $k_1$.
\end{thm}

The proof of this is done through a sequence of lemmas.

\begin{lemma}
  Let $k$ be a field for which \ref{eq:axiom1} holds, and $F$ a subfield that
  does not exclusively comprise constants of $k$. Let $\fn$ be the set of
  nontrivial divisors (???) $p$ of $F$ that are divisible by some $\pp$ of $\fm$.
  Then \ref{eq:axiom1} holds in $F$ for $\fn$.
\end{lemma}

\begin{lemma}
  Let $k$ be a field for which \ref{eq:axiom1} holds and $K/k$ a finite algebraic
  extension. Let $\fn$ be the set of all divisors $\fp$ of $K$ that divide some
  $\pp$ of $\fm$. Then \ref{eq:axiom1} holds in $K$ for some subset $\fn' \subset \fn$.
\end{lemma}

In fact, $\fn' = \fn$, but we do not demonstrate this here since it is not
required for the proof (the existence of any set of places satisfying our
criteria is sufficient).

\begin{lemma}
  \ref{eq:axiom1} holds in the case of $\bq$ and that of $K=k_1(z)$. The set $\fm$
  of valuations is the set of all valuations in the case of $\bq$, and the set of
  valuations trivial on $k_1$ in the latter case. The product formula is of the
  form

  \[ \prod_p {||a||}_p = 1 \]

  or a power, and there is no other relation between these valuations.

\end{lemma}

From the last two lemmas, we see that \ref{eq:axiom1} holds for the fields of
(??? thm2). The fact that all valuations of $\fm$ satisfy \ref{eq:axiom2}
follows from the fact that this is true in $\bq$ and hence in a finite extension
$k/\bq$.

It remains to prove the statement about the field $k_0$ of constants.

\begin{proof}
  Assume that $p$ is trivial on $k_1$. Then it will also be trivial on any
  algebraic extension of $k_1$.

  Hence we need only show that any element $a \in k_0$ is algebraic with respect
  to $k_1$. If $a$ were transcendental with respect to $k_1$, then from the
  minimal polynomial of $a$ over $k_1(z)$, we find that $z$ would have to be
  algebraic with respect to $k_1(a)$.

  Since $k_1(c)$ is in $k_0$, this would mean that $z$ is in $k_0$. This would
  imply that $k\subseteq k_0$, and $p$ would be trivial on $k$ -- which is a
  contradiction.
\end{proof}

\section{Characterizing fields by the product formula}
\label{sec:orgheadline28}

\subsection{(Main) theorem 3}
\label{sec:orgheadline27}
If a field satisfies \ref{subsec:axiom1} and \ref{sec:orgheadline20}, it is of one of the two types in \ref{sec:orgheadline25}. Furthermore, \ref{sec:orgheadline20} is satisfied for every place $\pp$.

\section{Parallelotopes}
\label{sec:orgheadline31}

\subsection{Theorem 4}
\label{sec:orgheadline29}
There are positive $C, D$ such that for all id\`eles $\fA$ we have
\[ CV(\fA) < M(\fA) \leq \max(1, DV(\fA)) \]


\subsection{Definitions}
\label{sec:orgheadline30}
Let $U$ be the multiplicative group of ``absolute units'', that is, $x\in k$ is in $U$ if ${||x||}_\pp = 1$ for all $\pp$.
\begin{itemize}
\item If there is a constant field $k_0$, $U=\ut {k_0}$.
\item ``In case order means number of elements, $U$ must be a finite group since it is contained in the parallelotope of size $1$, so $U$ consists of all roots of unity in $k$.
\end{itemize}
Now select a finite set $S$ of primes that contains all the archimedean primes. By $\fA_S$ we mean the id\`eles $\fA$ such that $\vval{ \fA } = 1$ for all $\pp\not\in S$. As one might expect, $e_\pp\in k$ which belong to $\fA_S$ are called $S$-units.
\end{document}
