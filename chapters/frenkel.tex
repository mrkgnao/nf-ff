\chapter{Adeles}

\section{Preliminaries from Galois theory}
We will let $K$ be a number field.
Denote by $\fldk$ the category of field extensions of $k$.

\begin{thm}[Fundamental theorem of Galois theory]
  There is a functor
  \[\dr \galfctk {\opp\fldk} \grp,\]
  the \textit{Galois group functor}.
\end{thm}

In particular, this means that given a $k$-automorphism $K\to L$, we get a
morphism of Galois groups
\[\gall \to \galk\]
since any automorphism of $L$ fixes $K$.

Recall that the field of \textit{cyclotomic numbers}, $\qzetn$, has Galois
group
\[\Gal(\qzetn/\bq) \cong \znzt \]
where $[n]\in\znz$ acts as the $n$-th power map.

\section{Class field theory}

\begin{thm}[Kronecker-Weber]
  The maximal abelian extension $\qab$ of $\bq$ satisfies
  \[ \qab = \bigcup_n \qzetn \]
  where, for $m|n$, we identify $\qzet m$ with the canonically given subfield of
  $\qzetn$.
\end{thm}

In particular, we may now apply $\gal - \bq$ to get the following:

\[ \agqab := \galqab \cong \varprojlim_n \znzt \]
 
Here the limit is taken with respect to the system of surjections

\[ \dr {\pi_ n^m} {\znzt} {\zmzt} \]

that sends, for instance, $[5]\in{\zmod 6}$ to $[1]\in\zmod 3$.

What does an element of $\agqab$ look like? By the definition of the inverse
limit of a filtered set (TODO check this), an element of $\agqab$ is a collection of
elements

\[ \alpha_n \in \znz \]

compatible with the $\pi_m^n$, where by
\textit{compatibility} we mean that

\[ m|n \implies \pi_n^m(\alpha_m) = \alpha_n. \]

\subsection{Describing $\agqab$ with $p$-adics}
(fill in defns of $\zp$ and $\qp$ later)

We have the following classical result:

\begin{thm}[Chinese remainder theorem]
  There exists an isomorphism
  \[ \znz \cong \prod_p \zmod {p^{\nu_p(n)}}. \]
\end{thm}

\begin{defn}
  We denote by
  \[ \zhat = \varprojlim_n \znz \]
  the \textit{profinite completion} of $\bz$, where the limit is taken with
  respect to the natural system of surjections considered in the previous section.
\end{defn}

Now note that

\begin{align*}
  \zhat &= \varprojlim_n \znz \\
        &\cong \varprojlim_n \prod_p \zmod {p^{\nu_p(n)}} \\
        &\cong \prod_p \varprojlim_r \zmod {p^r} 
\end{align*}

which finally gives us

\[ \zhat \cong \prod_p \zp. \]

Now observe that the Kronecker-Weber theorem can be understood as saying that
$\agqab \cong \ut{\zhat}$. Using the product expression for $\zhat$, we find
that

\[ \galqab \cong \prod_p \zpt. \]

\section{Class field theory}
The obvious next step is, given a number field $F/\bq$, to try to ``upgrade''
the Kronecker-Weber theorem and describe its maximal abelian extension $\ab F$.

No such analog is known. However, we do have a description of $\galfab$, the
abelianized Galois group of $F$, via class field theory.

\subsection{Adeles and ideles}
\label{sec:adeles-ideles}

\subsubsection{The special case of $F=\bq$}

Define the ring of \textit{integral adeles}

\[ \adz = \br \times \zhat \]

and the ring of \textit{adeles} as

\[ \adq = \adz \otimes_\bz \bq. \]

We can put a topology on this: let $\zhat$ have the product topology inherited
from the $\zp$, give $\bq$ the discrete topology, and let $\br$ have its usual
topology. This makes $\adq$ a topological ring, with a diagonal embedding
$\bq\to\adq$. There is a similar embedding $\qt \to \adqt$.

Notice that the quotient

\[ \adq/\bq \simeq \zhat \times (\br/\bz) \]

is compact, since $\zhat$ is a profinite group and hence compact.

In the special case of $F=\bq$, the statement of class field theory is that
$\galqab$ is isomorphic to the group of connected components of the quotient
$\adqt/\qt$. With the previous statement, we see that

\[ \adqt/
  \qt
  \simeq \rgt \times \prod_p \zpt. \]

Since $\rgt$ is very conencted, the group of connected components is isomorphic
to $\prod_p \zpt$, thus verifying the Kronecker-Weber theorem.

\subsubsection{More general settings}

Now we generalize to arbitrary $F/\bq$.


