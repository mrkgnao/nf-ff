\chapter{The structure of number fields}

We will follow \cite{frenkel} in this chapter. Throughout this chapter, when not
otherwise noted, $K$ and $F$ will be number fields.

\cite{kks1} and \cite{kks2} are references for the material in this chapter.

\section{Preliminaries from Galois theory}
Denote by $\fldk$ the category of field extensions of $k$.

\begin{thm}[Fundamental theorem of Galois theory]
  There is a functor
  \[\dr \galfctk {\opp\fldk} \grp,\]
  the \textit{Galois group functor}.
\end{thm}

Translated from the categorical language, there is an association

\[ K/k \longleftrightarrow \galk \]

for every field extension $K/k$ (or for every field $K$ ``over $k$''). This
association is \textit{functorial} in the sense that, for every $k$-morphism of
fields $\phi: K/k \to L/k$, we have a commutative diagram

\[
  \xymatrix{
    K/k   \ar[d] \ar[r]^\phi &L/k \ar[d]\\
    \galk &\gall \ar[l]^{\Gal(\phi)}   }
\]

\subsection{Cyclotomic fields}

Recall that the field of \textit{cyclotomic numbers}, $\qzetn$, has Galois
group
\[\Gal(\qzetn/\bq) \cong \znzt \]
where $[n]\in\znz$ acts as the $n$-th power map. This can be seen by noticing
that the action of an element $\sigma$ of the Galois group on the field is completely
determined by where it sends $\zeta_n$.

\section{Absolute Galois groups}

\subsection{Abelian extensions}

The study of general extensions of $\bq$ has proven technically challenging so
far. While the fact that $\bq$ is a ``nice'' field means we have Galois theory
at our disposal, arbitrary Galois extensions are too varied to be understood
using field-theoretic methods alone.

However, the \textit{abelian} extensions are a particularly tractable class
of extensions, whose structure is completely given by the class field theory of
number fields, or more generally, the class field theory of global fields (of
which function fields are the other main example).

\begin{defn}
  An \textit{abelian} extension of $\bq$ is an extension $K/\bq$ with
  $\Gal(K/\bq)$ abelian.
\end{defn}

For instance, extensions like $\bq(i)$ are manifestly abelian (the Galois group
is $\zmod 2$ in this case).

\begin{defn}
  The \textit{maximal abelian extension} of a field is the largest extension
  that has an abelian Galois group.
\end{defn}

One of the crowning achievements of algebraic number theory is the following
result, which shows us that $\qab$ can be understood in terms of ``simple''
extensions of $\bq$.

\begin{thm}[Kronecker-Weber]
  The maximal abelian extension $\qab$ of $\bq$ satisfies
  \[ \qab = \bigcup_n \qzetn \]
  where, for $m|n$, we identify $\qzet m$ with the canonically given subfield of
  $\qzetn$.
\end{thm}

\subsection{A first look at $\agqab$}

The Galois group of the maximal abelian extension, $\agqab$, is in fact the
abelianization of the absolute Galois group $\galq$ of $\bq$.
% \footnote{This holds for any field. (nice enough?)}
The Kronecker-Weber theorem allows us to apply $\gal - \bq$ to get the following
first description of the abelianized Galois group of $\bq$, noting that the
union operation is the colimit in the category of sets, and contravariant
functors like $\Gal(-/\bq)$ ``turn the arrows around'':

\[ \agqab := \galqab \cong \varprojlim_n \znzt \]

Here the limit is taken with respect to the system of surjections

\[ \dr {\pi_ n^m} {\znzt} {\zmzt} \text{ for all } m|n \]

that sends, for instance, $[5]\in{\zmod 6}$ to $[1]\in\zmod 3$. This can be
thought of as ``throwing away'' the information carried by the other factors
(which is just $2$ in this case).

What does an element of $\agqab$ look like? By the definition of the inverse
limit of a filtered set, an element of $\agqab$ is a collection of
elements

\[ \alpha_n \in \znz \]

compatible with the $\pi_m^n$, where by
\textit{compatibility} we mean that

\[ m|n \implies \pi_n^m(\alpha_m) = \alpha_n. \]

\subsection{$p$-adics}
We briefly outline the construction of the $p$-adic fields $\qp$.

Consider the rings $\zmod {p^n}$. In a way similar to what was done above (in
fact, this is a special case), we can define projections

\[ \dr {\pi_ s^t} {\zmod {p^s}} {\zmod {p^t}} \text{ for all } m|n \]

We can then construct the inverse limit:

\[ \bz_p = \varprojlim_r \zmod {p^r} \]

There is an embedding $\bz\hookrightarrow\bz_p$, by ``taking successive
remainders'': define the sequence

\[ t_n = t \Mod{p^n} \]

Then each such sequence $s(t) = (t_1, t_2, \cdots)$ is a $p$-adic integer.

\begin{example}
  $35$ embeds\footcite{Wikipedia} into $\bz_2$ as the sequence

  \[  (1, 3, 3, 3, 3, 35, 35, 35, \ldots)\]

  which, like all $p$-adics derived from positive integers, eventually ``stabilizes'' when
  the modulus becomes large enough.
\end{example}

We then define

\[ \qp = \rm{Frac}(\zp) \]

which we can do since $\zp$ is an integral domain.

Algebraic number theory books treat this construction in more detail. For
instance, chapter II of \cite{neukirch} is an excellent reference.

\subsection{Describing $\agqab$ with $p$-adics}

We have the following classical result:

\begin{thm}[Chinese remainder theorem]
  There exists an isomorphism
  \[ \znz \cong \prod_p \zmod {p^{\nu_p(n)}}. \]
\end{thm}

\begin{defn}
  We denote by
  \[ \zhat = \varprojlim_n \znz \]
  the \textit{profinite completion} of $\bz$, where the limit is taken with
  respect to the natural system of surjections considered in the previous section.
\end{defn}

Now note that

\begin{align*}
  \zhat &= \varprojlim_n \znz \\
        &\cong \varprojlim_n \prod_p \zmod {p^{\nu_p(n)}} \\
        &\cong \prod_p \varprojlim_r \zmod {p^r}
\end{align*}

which finally gives us

\[ \zhat \cong \prod_p \zp. \]

Now observe that the Kronecker-Weber theorem can be understood as saying that
$\agqab \cong \ut{\zhat}$. Using the product expression for $\zhat$, we find
that

\[ \galqab \cong \prod_p \zpt. \]

% \section{Class field theory}
% The obvious next step is, given a number field $F/\bq$, to try to ``upgrade''
% the Kronecker-Weber theorem and describe its maximal abelian extension $\ab F$.

% No such analog is known. However, we do have a description of $\galfab$, the
% abelianized Galois group of $F$, via class field theory.

\subsection{Ad\`eles }
\label{sec:ad\`eles}

Define the ring of \textit{integral ad\`eles}

\[ \adz = \br \times \zhat \]

and the ring of \textit{ad\`eles} as

\[ \adq = \adz \otimes_\bz \bq. \]

We can put a topology on this: let $\zhat$ have the product topology inherited
from the $\zp$, give $\bq$ the discrete topology, and let $\br$ have its usual
topology. This makes $\adq$ a topological ring, with a diagonal embedding
$\bq\hookrightarrow\adq$. There is a similar embedding $\qt \hookrightarrow \adqt$.

Notice that the quotient

\[ \adq/\bq \simeq \zhat \times (\br/\bz) \]

is compact, since $\zhat$ is a profinite group and hence compact.

In the special case of $F=\bq$, the statement of class field theory is that
$\galqab$ is isomorphic to the group of connected components of the quotient
$\adqt/\qt$. With the previous statement, we see that

\[ \adqt/
  \qt
  \simeq \rgt \times \prod_p \zpt. \]

Since $\rgt$ is very connected, the group of connected components is isomorphic
to $\prod_p \zpt$, thus verifying the Kronecker-Weber theorem.

Ad\`eles are related to \textit{id\`eles}, which are a multiplicative analogue that
we make use of in the next chapter.

% \subsubsection{More general settings}

% Now we generalize to arbitrary $F/\bq$.
